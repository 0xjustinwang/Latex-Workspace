\documentclass[11pt]{article}

% ----------- Font Packages ------------
\usepackage[T1]{fontenc}
\usepackage[utf8]{inputenc}

% ----------- Math Formatting ----------
\usepackage{siunitx}  
\AtBeginDocument{\RenewCommandCopy\qty\SI}       % Units like \SI{9.8}{m/s^2}
\usepackage{physics} 
\usepackage{amsmath, amssymb, amsthm}  % Core math packages
\usepackage{dsfont}
% Boxed theorem environments
\usepackage[most]{tcolorbox}
\tcbuselibrary{theorems}

% Cross-referencing and hyperlinks
\usepackage{hyperref}
\usepackage{cleveref}

% ----------- Layout and Spacing -------
\usepackage[letterpaper, top=0.75in, bottom=0.75in, left=1in, right=1in, heightrounded]{geometry}
\usepackage{setspace}
\onehalfspacing                % More readable line spacing for notes

% ----------- Math Formatting ----------
\usepackage{mathtools}        % Enhanced math commands
\usepackage{bm}               % Bold math symbols
\usepackage{cancel}           % Cross out terms in equations

% ----------- Misc Enhancements --------
\usepackage[protrusion=true, expansion=true, tracking=true, kerning=true, spacing=true, factor=1100, stretch=20, shrink=20, final, babel]{microtype} % Better font spacing
% \microtypesetup{disable=footnote} % Disable footnote patch to avoid errors
\usepackage{enumitem}         % Customizable lists
\usepackage{fancyhdr}         % Page headers/footers
\pagestyle{fancy}
\fancyhead[L]{CS Notes}

\usepackage{bookmark}         % Handle rerunfilecheck warnings

\hypersetup{
    colorlinks=true,
    linkcolor=blue,
    urlcolor=blue,
    citecolor=blue
}

% Optional: TikZ for diagrams
\usepackage{tikz}
\usetikzlibrary{arrows.meta, calc, decorations.markings}

% --------- Boxed theorem and definition environments ---------
\newtcbtheorem[number within=section]{theorem}{Theorem}%
{colback=white!97!blue!10,
 colframe=blue!50!black,
 fonttitle=\bfseries,
 coltitle=black,
 boxrule=0.5pt,
 arc=4pt,
 top=6pt,
 bottom=6pt,
 left=6pt,
 right=6pt,
 enhanced,
 attach boxed title to top left={yshift=-2mm,xshift=5mm},
 boxed title style={colframe=blue!50!black, colback=white}}{th}

\newtcbtheorem[number within=section]{definition}{Definition}%
{colback=white!95!green!10,
 colframe=green!50!black,
 fonttitle=\bfseries,
 coltitle=black,
 boxrule=0.5pt,
 arc=4pt,
 top=6pt,
 bottom=6pt,
 left=6pt,
 right=6pt,
 enhanced,
 attach boxed title to top left={yshift=-2mm,xshift=5mm},
 boxed title style={colframe=green!50!black, colback=white}}{def}



%  Document Start

\title{Lecture Notes: CMSC 14100 - Introduction to Computer Science}
 \author{Justin Wang}
\date{Autumn Quarter 2025}

\begin{document}

\maketitle

\section{Computational Thinking: 9/29/25}
\begin{itemize}
    \item \textbf{Decomposition/Abstraction:} Breaking down complex problems into component aspects, finding broadly relevant aspects where existing programs can be applied.
    \item \textbf{Modeling:} Creating simplified representations of complex systems to facilitate understanding and problem-solving.
    \item  \textbf{Algorithms:} Process that can be defined in a computer program.
    \item \textbf{Complexity:} Considering the amount of resources the computing would take. \begin{itemize}
        \item \textit{Time Complexity:} How long it takes to run the algorithm.
        \item \textit{Space Complexity:} How much memory it takes to run the
         \end{itemize}
\end{itemize}

\section{Programming Basics: 10/1/25}

\subsection{What is programming?}
\begin{center}
    "a collection of specifications that may take variable inputs, and that can be executed to produce outputs"
\end{center}

\section{Variables, Types, and Expressions: 10/3/25}
\subsection{How do computer's work?}
\begin{itemize}
    \item Processor: takes input values and operations, and produces output values.
    \item Memory: stores values for later retrieval.
\end{itemize}
\newpage
\subsection{Variables}
Let's say that we are trying to convert Fahrenheit to Celsius.
\begin{itemize}
    \item We can use variables to store values. For example, we can use the variable \texttt{fahrenheit} to store the temperature in Fahrenheit.
    \item We can then use the variable \texttt{celsius} to store the temperature in Celsius.
    \item We can then use the formula to convert Fahrenheit to Celsius: 
    \begin{center}
    \item $C = \frac{5}{9}(F - 32)$
    \end{center}
    where C is the temperature in Celsius and F is the temperature in Fahrenheit.
\end{itemize}
In python, we can write this as:
\begin{verbatim}
fahrenheit = 100
celsius = (5/9) * (fahrenheit - 32)
print(celsius)
\end{verbatim}

\subsection{Types}
\begin{itemize}
    \item numbers
    \begin{itemize}
        \item integers: whole numbers, e.g. -2, -1, 0, 1, 2
        \item floats: decimal numbers, e.g. -2.5, -1.0, 0.0, 1.0, 2.5
    \end{itemize}
    \item strings: text, e.g. "hello", "world"
    \item booleans: true or false, e.g. True, False
\end{itemize}
\textbf{Type Casting:} \\
Notice that "13" + 12 $\longrightarrow$ Error.
"13" + "12" = "1312" \\
So to actually add these as integer values:
\begin{verbatim}
int("13") + 12 = 25
\end{verbatim}
\subsection{Boolean Values and Logical Operators}
\begin{itemize}
    \item Boolean values: True, False
    \item Logical operators: and, or, not
    \item Comparison operators: ==, !=, <, >, <=, >=
\end{itemize}

\section{Statements and Functions: 10/6/25}
\begin{itemize}
    \item Statement: a command that tells your computer to perform an action.
    \item Function: takes in an input and produces an output.
\end{itemize}

\textbf{The Programming Cycle:}
\begin{enumerate}
    \item Understand the problem.
    \item Plan solution.
    \item Write code.
    \item Compile and run code.
    \item Test the code.
    \item Debug the code.
\end{enumerate}

\textbf{Programming Cycle for Functions:}
\begin{enumerate}
    \item Understand the problem.
    \item Consider the inputs and outputs for the function.
    \item Design what the function will do.
    \item Implement the function.
    \item Test the function.
\end{enumerate}
\section{Conditionals 10/8/25}
\textbf{Conditional Statements:}
\begin{itemize}
    \item if statement: executes a block of code if a condition is true.
    \item else statement: executes a block of code if the condition is false.
    \item elif statement: executes a block of code if the condition is true, and another block of code if the condition is false.
\end{itemize}
\textbf{In Python:}
\begin{verbatim}
    if (condition):
        statement
    ex: Feet to Inches:
    if (feet>0):
        return feet * 12
    if (feet<=0):
        return None
\end{verbatim}
\textbf{Scope:} The part of the program in which a variable is defined and accessible. 
\begin{center}
\textbf{"Rules" for scope:}
\end{center}
\begin{enumerate}
    \item Variables cannot be used before being declared.
    \item Variables declared in "outer" blcoks can be used in nested blcoks.
    \item Variables declared in "inner" blocks cannot be used in outer blocks.
\end{enumerate}

\section{Loops and Lists: 10/10/25}

\textbf{Loops:}
\begin{itemize}
    \item while loop: repeats a block of code while a condition is true.
    \item for loop: repeats a block of code for a specific number of times.
\end{itemize}
\textbf{Loop Structure:}
\begin{verbatim}
    for <variable> in <sequence>:
        <body>
\end{verbatim}
\textbf{Example: Print numbers from 1 to 10:}
\begin{verbatim}
    for i in range(1, 11):
        print(i)
\end{verbatim}
\textbf{Lists:} A structure that stores an arbitrarily large collection of data.
\begin{verbatim}
    my_list = [1,2,3,4,5,6,7,8,9,10]
\end{verbatim}
Index a list to iterate through each element in the list. 
\begin{verbatim}
    for num in my_list:
        print(num)
\end{verbatim}
\textbf{List Examples:}
\begin{itemize}
    \item A = []
    \item B = [1,2,3]
    \item C = ["hello", "world"]
    \item D = [1, "hello", 2.5, True]
\end{itemize}

\textbf{Example: Calculate Grades of Multiple Students:}
\begin{verbatim}
    scores = [17,30,100,99,52,1]
    def calculate_grades(scores):
        grades = []
        for score in scores:
            if score >= 90:
                grades.append("A")
            elif score >= 80:

                grades.append("B")
            elif score >= 70:
                grades.append("C")
            elif score >= 60:
                grades.append("D")
            else:
                grades.append("F")
        print("Done Grading")
        return grades
    print(calculate_grades(scores))
\end{verbatim}
\textbf{Executing a While Loop:}
Only proceed if condition is true:
\begin{verbatim}
    while (condition):
        <body>
\end{verbatim}
\textbf{Example: Print numbers from 1 to 10 using while loop:}
\begin{verbatim}
    i = 1
    while (i <=10):
        print(i)
        i += 1
\end{verbatim}
\textbf{More List Operations (10/13/25):}
\begin{itemize}
    \item Make list from a range:
\end{itemize}
\begin{verbatim}
    my_list = list(range(45)) # [0,1,2,...,44]
    print(my_list[24]) # 24
    print(my_list[-1]) # 44
    print(my_list[10:20]) # [10,11,...,19]
\end{verbatim}
Change values in a list:
\begin{verbatim}
    my_list[0] = 100
    print(my_list) # [100,1,2,...,44]
\end{verbatim}
Add values to a list:
\begin{verbatim}
    my_list.append(200)
    print(my_list) # [100,1,2,...,44,200]
    my_list.insert(0, -50) 
    print(my_list) # [-50,100,1,2,...,44,200]
\end{verbatim}
Remove values from a list:
\begin{verbatim}
    my_list.remove(100)
    print(my_list) # [-50,1,2,...,44,200]
    my_list.pop() # removes and returns last element
    print(my_list) # [-50,1,2,...,44]
    my_list.pop(0) # removes and returns element at index 0
    print(my_list) # [1,2,...,44]
    del my_list[0] # removes element at index 0
    print(my_list) # [2,3,...,44]
\end{verbatim}
Get length of a list:
\begin{verbatim}
    print(len(my_list)) # 43
\end{verbatim}
\textbf{String Review:}
\begin{itemize}
    \item String Delimitters: 
    \begin{verbatim}
    No return between items printed:
    print("Hello World", end="") 
    print("Bye")
    Output: Hello WorldBye
    \end{verbatim}
\end{itemize}
\textbf{Advanced Looping:} 
\begin{itemize}
    \item Accumulator:
\begin{verbatim}
    acc = 0
    list_of_numbers = [1,2,3,4,5]
    for i in list_of_numbers:
        acc += i
    print(acc) # 15
\end{verbatim}
    \item Continue Statement: skips the rest of the loop and goes to the next iteration.
\begin{verbatim}
    count_to_100 = list(range(101))
    for c in count_to_100:
        if (c % 2 ==0):
            print(c)
        else:
            continue
\end{verbatim}
    \item Break Statement: exits the loop entirely.
\begin{verbatim}
    inputs = [0, 2, 4, 6, 8, 10, -1, 10, 8, 6, 4, 2]
    for i in inputs:
        print(i)
        if i == -1:
            break
\end{verbatim}
\end{itemize}
\section{Tuples and More: 10/15/25}

\begin{verbatim}
# Check if two lists contain the same items (regardless of order)
list_a = ["Sports", "Physics", "CS"]
list_b = ["Physics", "Arts", "Music"]
similarities = 0

for i in list_a:
    for j in list_b:
        if i == j:
            print("item in a:", i)
            print("item in b:", j)
            similarities += 1

print(similarities)
\end{verbatim}

\subsection*{Tuples}

Tuples are ordered, immutable sequences that can store multiple items in one variable.

\subsection*{Creating and Accessing Tuples}

\begin{verbatim}
tup = (0, 1, 2)
print(tup[0])  # Access element 0

tup[0] = 10  # Error: tuples are immutable
\end{verbatim}

\subsection*{Returning Tuples from Functions}

\begin{verbatim}
numbers = [42, -7, 0, 19, 103, 58, -32, 7, 88]

def min_and_max(num_list):
    max_val = num_list[0]
    min_val = num_list[0]

    for n in num_list:
        if n > max_val:
            max_val = n
        if n < min_val:
            min_val = n

    return min_val, max_val  # returns a tuple
\end{verbatim}

\subsection*{Tuple Unpacking}

\begin{verbatim}
min_v, max_v = min_and_max(numbers)
print(min_v)
print(max_v)
\end{verbatim}

\subsection*{Unpacking in a Loop}

\begin{verbatim}
list_of_min_max = [
    min_and_max(numbers),
    min_and_max(numbers),
    min_and_max(numbers),
]

for m, n in list_of_min_max:
    print(m)
    print(n)
\end{verbatim}

\subsection*{Enumerate Function}

\texttt{enumerate()} returns pairs of \texttt{(index, value)} as tuples.

\begin{verbatim}
def min_and_max_index(num_list):
    max_val = num_list[0]
    max_index = 0
    min_val = num_list[0]
    min_index = 0

    for n in enumerate(num_list):  # e.g. [(0, 5), (1, 3), (2, 10)]
        if n[1] > max_val:
            max_val = n[1]
            max_index = n[0]
        if n[1] < min_val:
            min_val = n[1]
            min_index = n[0]

    return min_val, min_index, max_val, max_index
\end{verbatim}

\begin{verbatim}
min_v, min_i, max_v, max_i = min_and_max_index(numbers)
print((min_v, min_i, max_v, max_i))
\end{verbatim}

\subsection*{Reversed and Range}

\begin{verbatim}
for n in reversed(numbers):
    print(n)

# range(start, end, step)
for i in range(0, len(numbers), -1):  # backwards
    print(numbers[i])
\end{verbatim}

\subsection*{Strings}

\begin{verbatim}
name = "Nelson-Fromm"
print(name[0])  # returns "N"

fruit1 = "mango"
fruit2 = "kiwi"
fruit3 = "     Kiwi     "

print(fruit1 == fruit2)  # False
print(fruit3.lower().strip() == fruit2.lower())     # True (English only)
print(fruit3.casefold().strip() == fruit2.casefold())  # True (language-agnostic)
\end{verbatim}

\section{Strings and Lists of Lists: 10/18/25}

\subsection*{String Manipulation: }

\begin{verbatim}
full_name = "John Doe-Smith"
print(len(full_name)) # 15
list_from_my_name = full_name.split('-') 
print(list_from_my_name)  # ['John Doe', 'Smith']

#String Methods:
#.split(seperator/delimitter, 'maximum split -- optional')
#range(start, end, step -- optional)
print(full_name.split(' ')) # ['John', 'Doe-Smith']

lst = [list_from_my_name[0], list_from_my_name[1]]
print(lst)  # ['John Doe', 'Smith']
\end{verbatim}

\subsection*{Lists of Lists:}
\begin{verbatim}
    ascii_subset = [
        [65, 'A'],
        [66, 'B'],
        [67, 'C'],
        [68, 'D'],
        [69, 'E'],
    ]
print(ascii_subset[0])  # [65, 'A']
print(ascii_subset[0][1])  # 'A'
#ASCII uses:
encoding characters, send over networks, 
validating data, ciphers, alphabetical order

#ord()
print(ord('A'))  # 65

#chr()
print(chr(65))  # 'A'
\end{verbatim}

\subsection*{Slicing:}
\begin{verbatim}
my_list = [0,1,2,3,4,5,6,7,8,9]
print(my_list[2:5])  # [2,3,4]
print(my_list[:4])   # [0,1,2,3]
print(my_list[6:])   # [6,7,8,9]
print(my_list[::-1])  # [9,8,7,6,5,4,3,2,1,0]
print(my_list[:5:])  # [0,1,2,3,4]
#list[start:end:step]
\end{verbatim}

\section{More Lists of Lists 10/20/25:}
\begin{verbatim}

2d_list = [[0,0,0],[0,0,0],[]]
2d_list.append()

\end{verbatim}

\begin{itemize}

\item Think of the 2D List as a gameboard with rows and columns. 
\end{itemize}

\begin{verbatim}

list_2d = [ [0, 0, 0],
            [0, 0, 0],
            [0, 0, 0],
           ]

image = [[(r,g,b), (  ), (  )],
         [(  ), (  ), (  )],
         [(  ), (  ), (  )],
         [(  ), (  ), (  )]]
columns = nth item
rows = lists
height = 3
width = 4
pixel = image[0][0]
blue = pixel[2]

\end{verbatim}

\section{Function Execution: 10/22/25}
In the call stack:
\begin{itemize}
    \item Main: stores your global variables.
    \item Heap: an unorganized fill of values that maps to a variable, and is subject to change.
\end{itemize}
Pass by reference:
\begin{itemize}
    \item The "address" is given to the function as a parameter, so the mutable value at the address can be changed.
\end{itemize}
Pass by value:
\begin{itemize}
    \item Makes a copy of the function so modifying the arguments doesn't affect the variable globally.
\end{itemize}
Note: Mutable arguments get changed while immutable ones don't. \\
Deep Copy:
\begin{itemize}
    \item Creates a new copy where changes in the old does not affect the new.
\end{itemize}
Shallow Copy:
\begin{itemize}
    \item Creates a new list, or any mutuable item. but the elements of the list are references/addresses to the elements in the original list.
\end{itemize}





\end{document}