\documentclass[11pt]{article}
% -------------------------------------------------
% --------------- ENCODING & FONT ------------------
% -------------------------------------------------
\usepackage{stix2}
\usepackage[T1]{fontenc}
% -------------------------------------------------
% ---------------- PAGE LAYOUT --------------------
% -------------------------------------------------
\usepackage[letterpaper, top=0.8in, bottom=0.8in, left=1in, right=1in]{geometry}
\usepackage{setspace}
\onehalfspacing % Comfortable for reports
\usepackage{parskip} % No paragraph indent, space between paragraphs
\usepackage{fancyhdr}
\pagestyle{fancy}
\fancyhf{}
\fancyhead[L]{\textit{Lab 2: Avogadro's Number Derivation}}
\fancyhead[R]{\textit{\thepage}}
\renewcommand{\headrulewidth}{0.4pt}
% -------------------------------------------------
% ---------------- CHEMISTRY TOOLS ----------------
% -------------------------------------------------
\usepackage{siunitx} % For units and measurements
\sisetup{
 per-mode=symbol,
 separate-uncertainty=true,
 round-mode=places,
 round-precision=3
}
\usepackage[version=4]{mhchem} % For chemical equations and reactions
% Example: \ce{H2 + O2 -> H2O}
% -------------------------------------------------
% ---------------- MATH PACKAGES ------------------
% -------------------------------------------------
\usepackage{amsmath, amssymb, mathtools, bm}
\usepackage{cancel} % For striking out terms
\usepackage{physics} % Derivatives, vectors, etc.
% -------------------------------------------------
% ---------------- GRAPHICS & FIGURES -------------
% -------------------------------------------------
\usepackage{graphicx}
\usepackage{caption}
\usepackage{subcaption}
\usepackage{float} % Better figure placement
% -------------------------------------------------
% ----------------- TABLES ------------------------
% -------------------------------------------------
\usepackage{booktabs} % Beautiful tables
\usepackage{multirow}
\usepackage{array}
% -------------------------------------------------
% --------------- HYPERLINKS ----------------------
% -------------------------------------------------
\usepackage{hyperref}
\usepackage{bookmark}
\usepackage{cleveref}
\hypersetup{
 colorlinks=true,
 linkcolor=blue!50!black,
 urlcolor=blue!60!black,
 citecolor=blue!50!black,
 pdfauthor={Justin Wang},
 pdftitle={Chemistry Lab Report},
}
% -------------------------------------------------
% ----------------- DIAGRAMS ----------------------
% -------------------------------------------------
\usepackage{tikz}
\usetikzlibrary{arrows.meta, calc, decorations.markings, positioning}
% -------------------------------------------------
% ----------- CUSTOM BOXED ENVIRONMENTS ----------
% -------------------------------------------------
\usepackage[table]{xcolor}
% -------------------------------------------------
% ----------------- ENUMERATIONS ------------------
% -------------------------------------------------
\usepackage{enumitem}
\setlist{itemsep=4pt, topsep=4pt}
% -------------------------------------------------
% ----------------- TITLE FORMAT ------------------
% -------------------------------------------------
\usepackage{titlesec}
\titleformat{\section}{\large\bfseries}{\thesection.}{0.5em}{}
\titleformat{\subsection}{\normalsize\bfseries}{\thesubsection.}{0.5em}{}
% -------------------------------------------------
% ------------------ DOCUMENT INFO ----------------
% -------------------------------------------------
\title{\textbf{Lab 2: Avogadro's Number Derivation}}
\author{Justin Wang\\ University of Chicago \\ CHEM 11100}
\date{10/20/2025}
\begin{document}
\maketitle

\begin{center}
    {\small Lab Partner: Charlotte Zelin} \\
    {\small TA: Jordan Boysen}
\end{center}
\section*{Introduction}
\hspace{2em} The purpose of this experiment was to experimentally derive Avogadro's number by quantifying the number of
stearic acid molecules that are needed to form a monolayer on a surface of water. Avogadro's number is used as a conversion
factor between the number of moles and the number of particles in a substance, allowing chemists to do atomic-scale calculations 
with macroscopic quantities of material. Currently, as defined by the International System of Units (SI), Avogadro's number is exactly
$6.02214076 \times 10^{23}$ particles per mole, which is precisely the amount of atoms $12.0000$ grams of carbon-12 ($\ce{^{12}C}$). 
To obtain this number a plastic pipette was calibrated where the measurement of a graduated cylinder was practiced. Then by counting the drops 
of stearic acid solution needed to form a monolayer on the surface of water, the volume was calculated with dimensional analysis. 
Avogadro's number was then estimated with dimensional analysis and the known density and molar mass of stearic acid.

\section*{Experimental Procedure}
The procedure from the lab manual was followed, and three trials were conducted. 

\section*{Data Analysis}
\begin{table}[h!]
\centering
\begin{tabular}{ccccc}
\toprule
Trial & Watch Diameter (\si{\centi\meter}) & Drops/mL & mL/drop & Drops to monolayer (drops/mono) \\
\midrule
\#1 & 15.20 & 46 & .022 & 9 \\
\#2 & 15.19 & 47 & .021 & 7 \\
\#3 & 15.21 & 46 & .022 & 8 \\
\bottomrule
\end{tabular}
\caption{Measurements of watch glass diameters, drops per mL of n-hexane solution, mL of n-hexane solution per drop, and drops of stearic acid solution required to form a monolayer.}
\label{tab:watch_measurements}
\end{table}
\newpage
\subsection*{Sample Calculations}
{\small\textit{Disclaimer: All values used in the sample calculations will be truncated to four significant figures for readibility. Actual calculations were done with full precision,
and the final approximation of Avogadro's number will be rounded in the Discussion to reflect proper significant figures.}} 

Using data from Trial 3:

\begin{enumerate}[label=\arabic*.]
    \item \textbf{Volume of stearic acid solution to form monolayer:}
    \begin{align*}
        V_{solution} &= \frac{8 \text{ drops}}{46 \text{ drops/mL}} \\
        &= 0.1739 \text{ mL}
    \end{align*}
    \item \textbf{Mass of stearic acid added to the water surface:}
    \begin{align*}
        m_{stearic} &= V_{solution} \cdot \rho_{stearic} \\
        &= 0.1739\,\text{mL} \cdot \frac{1\,\text{L}}{1000\,\text{mL}} \cdot \left(\tfrac{0.148\,\text{g}}{\text{L}}\right) \\
        &= 2.574 \times 10^{-5} \text{ g}
    \end{align*}
    \item \textbf{Volume of the stearic acid monolayer:} 
    \begin{align*} 
        V_{mono} &= \frac{m_{stearic}}{\rho_{stearic}} \\[0.5em]   % adds a small vertical gap
        &= \frac{2.574 \times 10^{-5} \text{ g}}{0.941 \text{ g/cm}^3} \\
        &= 2.735 \times 10^{-5} \text{ cm}^3
    \end{align*}
    \item \textbf{Area of the water surface:}
    \begin{align*}
        A &= \frac{1}{4} \pi d^2 \\[0.5em]
        &= \frac{1}{4} \cdot \pi \cdot (15.21 \text{ cm})^2 \\[0.5em]
        &= 181.7 \text{ cm}^2 
    \end{align*}
    \item \textbf{Height of the stearic acid monolayer:}
    \begin{align*}
        h_{mono} &= \frac{V_{mono}}{A} \\[0.5em]
        &= \frac{2.735 \times 10^{-5} \text{ cm}^3}{181.7 \text{ cm}^2} \\
        &= 1.505 \times 10^{-7} \text{ cm}
    \end{align*}
    \item \textbf{Volume of a carbon atom:}
    \begin{align*}
        V_{atom} &= \left(\frac{h}{18}\right)^{\!3} \\
        &= \left(\frac{1.505\times 10^{-7} \text{ cm}}{18}\right)^{\!3} \\
        &= 5.850 \times 10^{-25} \text{ cm}^3
    \end{align*}
    \item \textbf{Molar volume of a carbon atom using diamond density:}
    \begin{align*}
        V_{mole} &= \frac{M_{C}}{\rho_{diamond}} \\[0.5em]
        &= \frac{12.01 \text{ g/mol}}{3.51 \text{ g/cm}^3} \\[0.5em]
        &= 3.422 \text{ cm}^3/\text{mol}
    \end{align*}
    \item \textbf{Avogadro's number:}
    \begin{align*}
        N_{A} &= \frac{V_{mole}}{V_{atom}} \\[0.5em]
        &= \frac{3.422 \text{ cm}^3/\text{mol}}{5.850 \times 10^{-25} \text{ cm}^3} \\[0.5em]
        &= 5.849 \times 10^{24} \text{ particles/mol}
    \end{align*}
\end{enumerate}
\begin{table}[h]
\centering
\begin{tabular}{cc}
\toprule
Trial & Avogadro's Number (\si{\per\mole}) \\ 
\midrule
1 & \num{4.09e24} \\ 
2 & \num{9.24e24} \\ 
3 & \num{5.85e24} \\ 
\midrule
Average & \num{6.39e24} \\ 
Accepted Value & \num{6.022e23} \\ 
Percent Error & \SI{962}{\percent} \\ 
\bottomrule
\end{tabular}
\caption{Calculation of Avogadro's number for three trials, and derivations for mean value and percent error.}
\label{tab:avogadro}
\end{table}
\newpage
 \section*{Discussion}
\hspace{2em} The experimentally determined value of Avogadro’s number was $6.39 \times 10^{24}$ particles per mole, obtained from the average of the three trials. This value exceeds 
the accepted constant of $6.022 \times 10^{23}$ particles per mole by approximately one order of magnitude, corresponding to a percent error of about \(962\%\).
The large discrepancy between the experimental and accepted values can be attributed to several assumptions made in the procedure for the sake of simplicity. First, it is noted in the manual 
that this method of determining Avogadro's number is designed to provide students an estimate within a power of ten of the accepted value, and thus the large error is somewhat expected. However,
in our experiment we edge slightly beyond this range. One source of error we can attribute to this was that we assumed the monolayer formed from the stearic acid solution had a uniform thickness equal to the length of a single stearic acid molecule. 
In reality, the monolayer likely had regions of varying thickness due to imperfections in the spreading of the stearic acid molecules. Another false assumption was the generalization of carbon's molar volume by using the density of diamond. Pure carbon consists of varying allotropes that all have 
different structures, and subsequently different densities. Perhaps the worst assumption made in this procedure was that the length of the stearic acid molecules consisted of 18 cubed carbon atoms forming a straight link. The structure of the stearic acid molecule
is more akin to a spheres of carbon atoms in a zig-zag shape. The error is compounded as we must exponentiate the cube by the power of three to get the estimated volume of a singular atom of stearic acid. 

\hspace{2em} This experiment, if repeated again, may benefit from a more precise pipetting technique. Rather than using a plastic pipette and estimating a milliliter of volume with drops, we could use a 1-10 mL micropipette to get a more precise volume delivery. Furthermore, estimating 
the length of stearic acid moleculues as spheres linked together may prove more accurate with the actual structure of the molecule. Finally, using a chemical marker to detect the monolayer may be a more precise indicator over waiting for a lens to form on top of the surface.








\end{document}