\documentclass[11pt]{article}

% ----------- Font Packages ------------
\usepackage[T1]{fontenc}
\usepackage[utf8]{inputenc}

% ----------- Math Formatting ----------
\usepackage{amsmath, amssymb, amsthm}
\usepackage{dsfont}
\usepackage[most]{tcolorbox}
\tcbuselibrary{theorems}

% Cross-referencing and hyperlinks
\usepackage{hyperref}
\usepackage{cleveref}

% ----------- Layout and Spacing -------
\usepackage[letterpaper, top=0.75in, bottom=0.75in, left=1in, right=1in, heightrounded]{geometry}
\usepackage{setspace}
\onehalfspacing

% ----------- Misc Enhancements --------
\usepackage{microtype}
\usepackage{enumitem}
\usepackage{fancyhdr}
\pagestyle{fancy}
\fancyhead[L]{Homework Two}

\usepackage{bookmark}

\hypersetup{
    colorlinks=true,
    linkcolor=blue,
    urlcolor=blue,
    citecolor=blue
}

\title{Algebra Homework: Scientific Notation and Exponents}
\author{Name: \underline{\hspace{3cm}}}
\date{Due: Nov 10}

\begin{document}

\maketitle

% ---------------------- Foundational ----------------------
\section*{Foundational Questions}

\begin{itemize}
    \item What is the purpose of writing numbers in scientific notation? Give two examples of when it’s useful in real life.
    \item What does it mean when an exponent is negative? How is \(10^{-3}\) related to \(10^3\)?
    \item Explain why multiplying by \(10^n\) moves the decimal point to the right, while dividing by \(10^n\) moves it to the left.
\end{itemize}

% ---------------------- Computation ----------------------
\section*{Exponents and Scientific Notation Practice}

Simplify and express your final answers in scientific notation where applicable. Show all work.

\begin{enumerate}[label=\textbf{(\arabic*)}, itemsep=1em]
    \item \(10^4 \times 10^3\)
    \item \(\dfrac{10^8}{10^5}\)
    \item \( (2^3)^4 \)
    \item \( 5^6 \div 5^2 \)
    \item \( (3^2)(3^4)(3^{-1}) \)
    \item Write \(45{,}000\) in scientific notation.
    \item Write \(0.00039\) in scientific notation.
    \item Simplify: \( (4 \times 10^5) \times (3 \times 10^2) \)
    \item Simplify: \( \dfrac{6.4 \times 10^{-3}}{8 \times 10^2} \)
    \item Convert to standard form: \(7.25 \times 10^{-4}\)
    \item Simplify completely: \(\dfrac{(2 \times 10^3)^2}{(8 \times 10^{-1})}\)
    \item \textbf{Challenge:} Simplify and write in scientific notation:
    \[
        \dfrac{(6.0 \times 10^{-5})(4.0 \times 10^{3})}{(1.2 \times 10^{-2})}
    \]
\end{enumerate}

% ---------------------- Word Problems ----------------------
\section*{Word Problems}

Show your work and express your answers in proper scientific notation.

\begin{enumerate}[label=\textbf{(\arabic*)}, itemsep=1.5em]
    \item The average distance from the Earth to the Sun is \(1.496 \times 10^8\) km.  
    The average distance from the Earth to the Moon is \(3.84 \times 10^5\) km.  
    How many times farther is the Sun than the Moon from Earth?

    \item A red blood cell has a diameter of about \(7.5 \times 10^{-6}\) meters.  
    A hair is about \(8 \times 10^{-5}\) meters thick.  
    Approximately how many red blood cells could fit across the width of one hair?

    \item The mass of an electron is \(9.11 \times 10^{-31}\) kg, and the mass of a proton is \(1.67 \times 10^{-27}\) kg.  
    How many times heavier is the proton than the electron?

    \item The Sun produces about \(3.8 \times 10^{26}\) joules of energy each second.  
    How much energy does it produce in one day? (Hint: there are 86,400 seconds in a day.)

    \item A bacterium doubles its population every hour. If there are initially \(4.0 \times 10^3\) bacteria, write an expression for the population after \(t\) hours. Then find the population after 6 hours.

    \item \textbf{Challenge Problem}

    The speed of light is \(3.0 \times 10^8 \text{ m/s}\).  
    A light-year is the distance light travels in one year. There are approximately \(3.15 \times 10^7\) seconds in one year.

    \begin{enumerate}[label=\textbf{(\alph*)}]
        \item Write an expression for one light-year in meters and simplify.
        \item Express your result in scientific notation.
        \item If a star is \(4.2\) light-years away, how far is it in meters?
    \end{enumerate}
\end{enumerate}

\bigskip

\noindent\textbf{Extra Practice (optional):} Simplify each of the following:
\begin{enumerate}[label=\textbf{(\alph*)}, itemsep=0.5em]
    \item \( (5^2 \times 5^{-4})^3 \)
    \item \( \dfrac{(2.5 \times 10^{-3})^2}{(5 \times 10^{-7})} \)
    \item \( \dfrac{1}{10^{-4}} \)
\end{enumerate}

\end{document}
