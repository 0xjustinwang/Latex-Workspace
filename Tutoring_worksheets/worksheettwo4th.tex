\documentclass[11pt]{article}

% ----------- Font Packages ------------
\usepackage[T1]{fontenc}
\usepackage[utf8]{inputenc}

% ----------- Math Formatting ----------
\usepackage{amsmath, amssymb, amsthm}
\usepackage{dsfont}
\usepackage[most]{tcolorbox}
\tcbuselibrary{theorems}

% Cross-referencing and hyperlinks
\usepackage{hyperref}
\usepackage{cleveref}

% ----------- Layout and Spacing -------
\usepackage[letterpaper, top=0.75in, bottom=0.75in, left=1in, right=1in, heightrounded]{geometry}
\usepackage{setspace}
\onehalfspacing

% ----------- Misc Enhancements --------
\usepackage{microtype}
\usepackage{enumitem}
\usepackage{fancyhdr}
\pagestyle{fancy}
\fancyhead[L]{Homework Three}

\usepackage{bookmark}

\hypersetup{
    colorlinks=true,
    linkcolor=blue,
    urlcolor=blue,
    citecolor=blue
}

\title{Algebra Homework: Exponents and the Order of Operations (PEMDAS)}
\author{Name: \underline{\hspace{3cm}}}
\date{Due: Nov 17}

\begin{document}

\maketitle

% ---------------------- Foundational ----------------------
\section*{Foundational Questions}

\begin{itemize}
    \item What does an exponent tell you to do? Give two examples using both positive and negative exponents.
    \item What is the value of any nonzero number raised to the power of zero? Explain why this rule makes sense.
    \item In the expression \(8 + 3 \times 2^2\), which operation do you do first? Why does the order of operations matter?
    \item Write out what each letter in PEMDAS stands for, and explain the difference between multiplication/division and addition/subtraction in the order hierarchy.
\end{itemize}

% ---------------------- Computation ----------------------
\section*{Exponent Rules and Order of Operations}

Simplify each expression. Show every step in the correct order (PEMDAS).

\begin{enumerate}[label=\textbf{(\arabic*)}, itemsep=1em]
    \item \(2^4\)
    \item \(5^0\)
    \item \(3^{-2}\)
    \item \(4^3 \times 4^2\)
    \item \(\dfrac{8^6}{8^4}\)
    \item \((2^3)^2\)
    \item \((5 \times 10)^2\)
    \item \(6^2 + 3 \times 2^3\)
    \item \((7 + 3)^2 - 4^2\)
    \item \(12 \div (3 \times 2)^2\)
    \item \([2 + 3^2] \times (4 - 1)\)
    \item \(5 + 2 \times (3 + 4)^2\)
    \item \(\dfrac{(2^3 + 4)^2}{8}\)
    \item \(6 - [3^2 - (2 + 1)]\)
    \item \(\dfrac{4^3 - 2^4}{2^2}\)
    \item \textbf{Challenge: } Simplify step by step:
    \[
        \dfrac{(3 + 2^3)^2 - (4^2 + 1)}{(2^2)^2}
    \]
\end{enumerate}

% ---------------------- Word Problems ----------------------
\section*{Word Problems and Applications}

Show your work. Use order of operations carefully when translating the problem into math.

\begin{enumerate}[label=\textbf{(\arabic*)}, itemsep=1.5em]
    \item A square field has a side length of \(12\) meters.  
    Write and compute the expression for its area using exponents.

    \item A cube has side length \(5\) cm.  
    Write an expression for its volume and evaluate.

    \item An online video gets twice as many views each day as the day before.  
    If it starts with \(1{,}000\) views, write an expression for the number of views after 4 days and find the result.

    \item A scientist observes a bacteria colony that triples every 2 hours.  
    Starting with \(200\) bacteria, how many are there after 6 hours? (Hint: use exponents.)

    \item A store sells packs of batteries for \$4 each, plus a one-time fee of \$3 for shipping.  
    Write an expression using parentheses that correctly gives the cost of buying 5 packs, and evaluate it.

    \item \textbf{Challenge Problem}

    A rocket’s height (in meters) after \(t\) seconds is modeled by:
    \[
        h(t) = 5t^2 + 10t + 25
    \]
    \begin{enumerate}[label=\textbf{(\alph*)}]
        \item Find the rocket’s height after \(t = 4\) seconds.
        \item What is the first term of the expression responsible for the rocket’s acceleration? Why?
        \item If the rocket’s initial height doubles, how would you modify the equation? Write the new form.
    \end{enumerate}
\end{enumerate}

% ---------------------- Extra Practice ----------------------
\bigskip
\noindent\textbf{Extra Practice (optional):} Simplify each of the following using PEMDAS:
\begin{enumerate}[label=\textbf{(\alph*)}, itemsep=0.5em]
    \item \(3^2 + 4^2\)
    \item \(10 - 2 \times (3^2 + 1)\)
    \item \((2^3 + 3^2) \div 5\)
    \item \(2[(4 + 2)^2 - 3^2]\)
\end{enumerate}

\end{document}
